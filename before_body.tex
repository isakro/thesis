\pagestyle{plain}
\pagenumbering{roman}
\setcounter{page}{3}

\section*{Acknowledgements}
In bringing this project to a conclusion I would first like to express my profound gratitude to my three excellent supervisors David Wright, Anders Romundset and Ingrid Fuglestvedt. David has by some mysterious ability allowed me both ample creative room to aimlessly explore interesting albeit not always as fruitful ideas, while at the same time making sure to keep me firmly on track. David's effectiveness, thoroughness, extensive knowledge of the discipline, and help with manouvering small and large peculiarities of the academic world has formed a throughly dependable foundation on which to undertake this project. Anders has been cruical for all my dealings with geology through the project, and was instrumental for productively relating relative sea-level change to the archaeologcal record. Anders has shared generously and pedagogically of his time and knowledge when reading texts, during two rounds of fieldwork in the Oslo region and when showing me the ropes of the subsequent laboratory work during my one week visit to the Geological Survey in Trondheim. Finally, Ingrid has read my texts with great enthusiasm and encouragement, and her overall curiosity and enthusiasm for seemingly all aspects of archaeology and related disciplines has been very motivational, whether shared through supervision or informal talks in the break-room.  

I would also like to thank the different research groups that I have been be a part of throughout the project. This includes Prehcoast and Archaeology by Proxy at the Museum of Cultural History, and the Materialities group at the Institute of Archaeology, Conservation and History. Through these I have been able to present and discuss my work with a great range of archaeologists, students of archaeology, and non-archaeologists, both formally and informally -- offerring both welcome delves and escapes from the depths of Mesolithic and computational archaeology.

Steinar Solheim, who is also co-author for the final paper of this thesis, has provided valuable feedback and continous encouragement both in the time before and throughout the project. Per Persson has also shared extensively from his own critical engagement with shoreline dating and his wide knowledge of Scandianavian Stone Age archaeology more generally. I am also very grateful to Daniel Gross who evaluated my work for my midway assessment. Daniel's input came at the perfect time, and although all of his suggestions did not end up being directly integrated in the final product, all are represented in the thesis in one way or another.

The disciplinary and administrative support of the PhD-programme offered by the institute has also been excellent. The courses offered through DialPast have been invaluable, and the help I got with both finding and participating in courses offered by other institutions . The PhD seminars led by Martin Fürholt, Christopher Prescott and Knut Ivar Austvoll were also very stimulating, and  . Knut Ivar also deserves thanks for borrowing me extra computer power towards the end of the project, but especially for being.  DigDok at the Museum of Cultural History also deserve great thanks for help with accessing and retrieving data used in the thesis.

Finally, I'm grateful to all the people at IAKH and at Blindernveien 11 for making it a stimulating and great place to have spent most of my days over the last three years, and to my fellow PhD students, and especially Hallvard Bruvoll both at the institute and the museum. Especially Hallvard Bruvoll who started one month before me and who has also been contenting with the 

\section*{Summary}
Through a series of case-studies on the Skagerrak coast of south-eastern Norway, this thesis is concerned with contributing to getting a handle on the vast amounts of data that have been and continue to be generated by Mesolithic archaeology in Norway. Fundamental questions concerning the quantity, composition and chronological control we have of the material available to us dictates not only our understanding of the period, but also determines what questions we can hope to answer. This follows from the resolution and quality of the archaeological record, and the varying spatial and temporal scales that different behavioural and societal dynamics operate over. The degree to which the characteristics of our data matches those necessary to illumnate these dynamics thus informs the kind of archaeology we can reliably hope to do.   

However, given that this data is neither recorded or can be approached without a predefined understanding of what dimensions of it are of interest, the material was explored using a series of heuristic models that are frequently drawn on in the literature to understand prehistoric hunter-gatherer societies and the coastal Mesolithic of Northern Europe. These pertain especially to the dating of the archaeological material, patterns of land-use and mobility, as well as general trends of demographic dynamics. 

The thesis consists of four papers and an introductory text. The introductory text establishes the context and motivation for undertaking the four studies, relates them to each other, and presents further avenues along which the results can be explored, extended, and potentially be substantiated further. With an aim to make these efforts transparent, reproducible and extendable, the entirety of the project has been undertaken within a framework of open science. This means that all text, figures, data and code is made freely available for anyone to access and scrutinise. Along the same lines, and in part following from the quantiative nature of the derived results, all substantive inferences are instatiated as explicit causal hypotheses in the introductory text. The purpose of this is to promote clear routes for critical engagement and further research efforts.  

Some central results follows from a quantitative assessment of the relationship between Stone Age sites and the contemporenous shoreline in the region,  which largerly verified the previously suggested tendency of the sites to have been located close to the shoreline when they were occupied. Following from the quantiative nature of this assessment, this resulted in the development of a new method for shoreline dating Stone Age sites in the region based on their altitude relative to the present-day sea-level. This method for shoreline dating has been made available as a freely available package for the R programming language, making the method readily accessible for anyone to apply and assess. 

The second part of the thesis focused on the composition of lithics assemblages. These were analysed by employing multivariate statistics to structure the analysis. This was done both to assess the development in the occurence of artefact types in established use in Norwegian archaeology, largely verifying the general trends of the current understanding of these developments. However, a more novel insight follows from the identification of potential relevance of. Volumetric density of lithics and the proportion of lithics that have been subjected to secondary modification have been argued to be a reflection of land-use and mobility patterns. While in need of further substantiation and evaluation, these developments in these dimensions correspond to that have previously

The final part of the analysis 


\tableofcontents

\clearpage
\addcontentsline{toc}{chapter}{\listfigurename}
\listoffigures

\clearpage
\addcontentsline{toc}{chapter}{\listtablename}
\listoftables
\clearpage
\pagestyle{plain}

\pagestyle{fancy}
\fancyhead{} % Clear all header fields
\fancyfoot{} % Clear all footer fields
\fancyhead[LE,RO]{\thepage} % Page number on the left for even pages, right for odd pages
\fancyhead[RE]{\textit{Chapter \thechapter. \leftmark}} % Chapter title on the right for even pages
\fancyhead[LO]{\textit{\rightmark}} % Section title on the left for odd pages
\renewcommand{\chaptermark}[1]{\markboth{#1}{}} % Customize chapter mark
\renewcommand{\sectionmark}[1]{\markright{#1}} % Customize section mark
\renewcommand{\headrulewidth}{0.001pt}
\clearpage{\pagestyle{empty}\cleardoublepage}