\pagestyle{plain}
\pagenumbering{roman}
\setcounter{page}{3}

\section*{Acknowledgements}
In bringing this project to a conclusion I would first like to express my profound gratitude to my three excellent supervisors David Wright, Anders Romundset and Ingrid Fuglestvedt. David has by some mysterious ability allowed me both ample creative room to aimlessly explore interesting albeit not always as fruitful ideas, while at the same time making sure to keep me firmly on track. David's effectiveness, thoroughness, extensive knowledge of the discipline, and help with manouvering small and large peculiarities of the academic world has formed a throughly dependable foundation on which to undertake this project. Anders has been cruical for all my dealings with geology through the project, and was instrumental for productively relating relative sea-level change to the archaeologcal record. Anders has shared generously and pedagogically of his time and knowledge when reading texts, during two rounds of fieldwork in the Oslo region and when showing me the ropes of the subsequent laboratory work during my one week visit to the Geological Survey in Trondheim. Finally, Ingrid has read my texts with great enthusiasm and encouragement, and her overall curiosity and enthusiasm for seemingly all aspects of archaeology has been very motivational, whether shared through supervision or informal talks in the break-room.  

I would also like to thank the different research groups that I have been be a part of throughout the project. This includes Prehcoast and Archaeology by Proxy at the Museum of Cultural History, and the Materialities group at the Institute of Archaeology, Conservation and History. Through these I have been able to present and discuss my work with a great range of archaeologists, students of archaeology, and non-archaeologists, both formally and informally -- offerring both welcome delves and escapes from the depths of Mesolithic and computational archaeology.

Steinar Solheim, who is also co-author for the final paper of this thesis, has provided valuable feedback and continous encouragement both in the time before and throughout the project. Per Persson has also shared extensively from his own critical engagement with shoreline dating and his wide knowledge of Scandianavian Stone Age archaeology more generally. I am also very grateful to Daniel Groß who evaluated my work for my midway assessment. Daniel's input came at the perfect time, and although all of his suggestions did not end up being directly integrated in the final product, all are represented in the thesis in one way or another.

I'm grateful to all the people at IAKH and Blindernveien 11 for making it a stimulating and great place to have spent most of my days over the last three years, Partricularly my fellow PhD students, including those at the Museum of Cultural History,. Especially Hallvard Bruvoll who started one month before me and was the canary ahead of me in the coal mine.

Finally, the disciplinary and administrative support of the PhD-programme offered by the institute has also been excellent, and especially the courses offered through DialPast have been invaluable. The PhD seminars led by Martin Fürholt, Christopher Prescott and Knut Ivar Austvoll were also very stimulating. Knut Ivar deserves special thanks for borrowing me extra computer power towards the end of the project, but especially for being.  DigDok at the Museum of Cultural History also deserve great thanks for help with accessing and retrieving data used in the thesis.


\section*{Summary}
Through a series of case-studies that employ Mesolithic data from the Skagerrak coast of south-eastern Norway, this thesis is concerned with contributing to getting a handle on the vast amounts of data that have been and continue to be generated by Mesolithic archaeology in Norway. Fundamental questions concerning the quantity, composition and chronological control we have of the material available to us dictates not only our understanding of the period, but also determines what questions we can hope to answer. This follows from the resolution and quality of the archaeological record, and the varying spatial and temporal scales that different behavioural and societal dynamics operate over. The degree to which the characteristics of our data matches those necessary to illumnate these dynamics thus informs the kind of archaeology we can reliably hope to do.   

Given that this data is neither recorded or can be approached without a predefined understanding of what dimensions of it are of interest, the material was explored using a series of heuristic models that are frequently drawn on in the literature to understand prehistoric hunter-gatherer societies and the coastal Mesolithic of Northern Europe. These pertain especially to the dating of the archaeological material, patterns of land-use and mobility, as well as general trends of demographic development. 

The thesis consists of four papers and an introductory text. The introductory text establishes the context and motivation for undertaking the four studies, relates them to each other, and presents further avenues along which the results can be explored, extended, and potentially be substantiated further. With an aim to make these efforts transparent, reproducible and extendable, the entirety of the project has been undertaken within a framework of open science. This means that all text, figures, data and code is made freely available for anyone to access and scrutinise. Along the same lines, and in part following from the quantiative nature of the derived results, all substantive inferences are instatiated as explicit causal hypotheses in the introductory text. The purpose of this is to promote clear routes for critical engagement and further research efforts.  

Some central results from the papers of the thesis follows first from a quantitative assessment of the relationship between Stone Age sites and the contemporenous shoreline in the region, which largerly verified the previously suggested tendency of the sites to have been located close to the shoreline when they were occupied. Following from the quantiative nature of this assessment, this resulted in the development of a new method for shoreline dating Stone Age sites in the region based on their altitude relative to the present-day sea-level. This method for shoreline dating has been made available as a freely available package for the R programming language through the second paper of the thesis, making the method readily accessible for anyone to apply and assess. 

The third paper of the thesis focused on approaches for analysing the composition of a larger number of lithic assemblages by structuring the analysis through the use of multivariate statistics. First, this was done to assess the temporal development in the occurence of artefact types that are in established use in Norwegian archaeology, with the findins largely verifying the current understanding of these developments. More novel insights follows from the second line of investigation which explored dimensions of the assemblages that have frequently been drawn on ouside the Scandinavian setting to map variation in land-use and mobility patterns associated with the formation of lithic assemblages. Specifically, the volumetric density of lithics on the sites and the proportion of these that have been subjected to secondary modification were found to be negatively correlated. Furthermore, this was also found to follow a temporal development with an increase in the density of lithics and decrease in secondarily worked lithics over time. Drawing on the substantive implications that this relationship has been ascribed in other context, the findings could indicate a corresponding overall decrease in occupational duration at the sites through the period. While the relevance of this framework for the Scandinavian Mesolithic is in need of further substantiation and evaluation, these developments match those previously suggested in the literature concerned with the Mesolithic of south-eastern Norway, giving some additional support for its relevance as a quantitative measure for land-use and mobility patterns in the Scandinavian Mesolithic.  

By drawing on approaches that have been used for analogous treatment of radiocarbon dates, the final paper of the thesis developes a method for assessing the summed probability of multiple shoreline dates with the aim of mapping the frequency distribution of shoreline dated sites over time. This was then compared to an analysis of the summed probability of radiocarbon dates from within the same area, as both measures have previously been suggested to be related to demographic dynamics. The development of the two proxies diverged, with the frequncy of shoreline dated sites following some process of decline through the period and the frequency of radiocarbon dates commencing later than the shoreline dates and reaching a stable plateu after an initial period of growth. Consequently, while it seems reasonable that both measures hold some demographic signal, the divergence between them would mean that some effects are confounding this relationship. In the paper it was suggested that mobility patterns is the main determinant for the development in the frequency of shoreline dated sites over time. While this was also suggested to impact the radiocarbon dates, we propose that this is not as substantial as with the shoreline dates. However, properly understanding this relationship will depend on directly assessing the influence of such confounding effects


\tableofcontents

\clearpage
\addcontentsline{toc}{chapter}{\listfigurename}
\listoffigures

\clearpage
\addcontentsline{toc}{chapter}{\listtablename}
\listoftables
\clearpage
\pagestyle{plain}

\pagestyle{fancy}
\fancyhead{} % Clear all header fields
\fancyfoot{} % Clear all footer fields
\fancyhead[LE,RO]{\thepage} % Page number on the left for even pages, right for odd pages
\fancyhead[RE]{\textit{Chapter \thechapter. \leftmark}} % Chapter title on the right for even pages
\fancyhead[LO]{\textit{\rightmark}} % Section title on the left for odd pages
\renewcommand{\chaptermark}[1]{\markboth{#1}{}} % Customize chapter mark
\renewcommand{\sectionmark}[1]{\markright{#1}} % Customize section mark
\renewcommand{\headrulewidth}{0.001pt}
\clearpage{\pagestyle{empty}\cleardoublepage}